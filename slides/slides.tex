\documentclass[10pt]{beamer}

\usetheme{m}

\usepackage{booktabs}
\usepackage{graphicx}
\usepackage[scale=2]{ccicons}
\usepackage{pgfplots}
\usepgfplotslibrary{dateplot}

\title{Git Submodules}
\subtitle{An intro to managing dependencies}
\date{\today}
\author{Miguel Gonzalez and Jan Kerkenhoff}
\institute{Fontys Hogeschool Venlo}
% \titlegraphic{\hfill\includegraphics[height=1.5cm]{logo/logo}}

\begin{document}

\maketitle

\begin{frame}
  \frametitle{Table of Contents}
  \setbeamertemplate{section in toc}[sections numbered]
  \tableofcontents[hideallsubsections]
\end{frame}


\section{Introduction}
\begin{frame}
  \frametitle{Monoliths (really big projects!)}
  \begin{center}\includegraphics[width=280px]{images/monoliths.jpg}\end{center}
\end{frame}

\begin{frame}
  \frametitle{Monoliths (continued)}
  In terms of \textbf{git} there are a lot of drawbacks:
  \begin{itemize}
  	\item takes a long time to clone
  	\item many contributors
  	\item hard to keep track of all changes
  	\item wastes local disk space
  \end{itemize}  
\end{frame}

\begin{frame}
  \frametitle{the answer to everything: git-submodules!}
  \begin{center}\includegraphics[width=200px]{images/divide.jpg}\end{center}
  \begin{center}\textbf{git submodule }add $<$repository-url$>$\end{center}
\end{frame}

\section{Working on Submodules}

\begin{frame}[fragile]
  \frametitle{Getting upstream changes}
   Submodules require you to  \alert{explicitly} trigger an update.
   This protects you from blowing your project up with unexpected changes to one of the submodules.    
  \begin{center}cd $<$your-submodule-path$>$\\ \texttt{git submodule update --remote} \end{center}
  
  Git now automatically \emph{fetches} the changes from upstream and \emph{merges} them.

\end{frame}

\begin{frame}[fragile]
  \frametitle{Using different remotes than master}
  Git defaults to using the master branch to pull in changes when you update submodules.
  This can be changed by setting a different remote for the submodule. There a two places to change this:
\begin{center} 
  \texttt{.gitmodules} changed for everyone \\
  \texttt{.git/config} changed locally for just you
\end{center}
\end{frame}

\begin{frame}
\frametitle{Submodules start with a detached HEAD}
 \begin{center}
 \includegraphics[width=280px]{images/nixon.jpg}
 \end{center}
\end{frame}


\begin{frame}[fragile]
  \frametitle{Setting up submodules for changes}
   Submodules are normally in an detached head state, so that any changes are not tracked. To start tracking the changes we need to create a branch that contains the changes we do to our submodule.

\end{frame}

\begin{frame}
	\frametitle{digging deeper..}
	\begin{itemize}
		\item When initialising a submodule, a \textbf{.gitmodules} file will be created. It contains a \emph{SHA1} commit id.
		\item \textbf{git-grep} does not look for submodules.
		\item committing in a submodule does not update the \textbf{.gitmodules} file!
		\item push the committed \textbf{.gitmodules} file to update the submodule reference on origin.
	\end{itemize}
\end{frame}

\begin{frame}
	\frametitle{Sounds awesome, doesn't it?}
\begin{center}\includegraphics[width=280px]{images/goodbad.jpg}\end{center}
\end{frame}

\section{Pros and Cons}

\begin{frame}
  \frametitle{Pros (the strength)}
  	Submodules let divide your repository into smaller parts.
	\begin{itemize}
		\item each submodule can be cloned individually.
		\item pushed commits from a submodule go directly to the specified origin (of the submodule).
		\item easy to distinguish between different projects (divided into multiple repositories)
		\item no disk space waste, only file with pointers exist.
	\end{itemize}
\end{frame}

\begin{frame}
	\frametitle{Cons (well..)}
	Submodules lack of features and design consistencies.
	\begin{itemize}
		\item hard to get rid of obsolete submodules
		\item moving code between submodules can cause duplication, irritations and merge conflicts
		\item user has to manage modules via meta files like .gitmodules file
		\item required code must be cloned individually
	\end{itemize}
\end{frame}
   
\section{Alternative: subtrees}
\begin{frame}{Animation}
\frametitle{how do subtrees work}
	\begin{itemize}[<+- | alert@+>]
		\item Dependencies are added as new remote for a new branch
		\item Branch is read in to the main project via \textbf{git read-tree}
		\item Getting new changes is done via merging in the dependency branch
		\item \textbf{git diff tree} can be used in stead of \textbf{git diff}
	\end{itemize}

\end{frame}	

\end{document}