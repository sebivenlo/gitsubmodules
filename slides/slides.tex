\documentclass[10pt]{beamer}

\usetheme{m}

\usepackage{booktabs}
\usepackage[scale=2]{ccicons}

\usepackage{pgfplots}
\usepgfplotslibrary{dateplot}

\title{Git Submodules}
\subtitle{An intro to managing dependencies}
\date{\today}
\author{Miguel Gonzalez and Jan Kerkenhoff}
\institute{Fontys Hogeschool Venlo}
% \titlegraphic{\hfill\includegraphics[height=1.5cm]{logo/logo}}

\begin{document}

\maketitle

\begin{frame}
  \frametitle{Table of Contents}
  \setbeamertemplate{section in toc}[sections numbered]
  \tableofcontents[hideallsubsections]
\end{frame}


\section{Introduction}
\section{Pros and Cons}
\section{Live Demo}
\section{Updating Submodules}

\begin{frame}[fragile]
  \frametitle{Getting upstream changes}
   Submodules require you to  \alert{explicitly} trigger an update.
   This protects you from blowing your project up with unexpected changes to one of the submodules. The easiest way to update a submodule after entering the submodule directory is:

  \begin{center} \texttt{git submodule update --remote} \end{center}
  
  Git now automatically \emph{fetches} the changes from upstream and \emph{merges} them.

\end{frame}

\begin{frame}[fragile]
  \frametitle{Using different remotes than master}
  Git defaults to using the master branch to pull in changes when you update submodules.
  This can be changed by setting a different remote for the submodule. There a two places to change this:
\begin{center} 
  \texttt{.gitmodules} changed for everyone \\
  \texttt{.git/config} changed localy for just you
\end{center}

\end{frame}
\section{Working on Submodules}

\end{document}